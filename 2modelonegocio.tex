\chapter{Modelo del Negocio}
\section{Contexto}
La Franquicia de Farmacias "Fran Farmacias" se dedica a la venta de medicamentos que requieren o no receta. La franquicia consta con un único dueño y múltiples empleados atendiendo las sucursales, de la misma manera el supervisor de la sucursal en la que esta asignado, verifica en la apertura y al cierre del turno que la operación de la sucursal sea de la mejor manera posible y lo mismo pasa en las diferentes sucursales de la franquicia\\

\section{Términos del Negocio}

Cliente: Se refiere a todas las personas físicas y morales que Compran medicamentos ya sea que estos sean clientes registrados en el sistema o sin registrar.\\

Cliente Preferente: Se refiere a todas aquellas personas físicas y morales que Compran medicamentos y que están registrados en el sistema, estos clientes cuentan con un monedero electrónico



Dosis: Cantidad a ingerir o suministrar expresado en unidades de volumen o peso(gramos) por unidad de un Medicamento.\\

Dueño: Es propietario de la farmacia y de todas sus sucursales, se encarga de la contratación de todos los Empleados, es el Empleado con mayor rango en “Fran Farmacias”.\\

Empleado: Se refiere a cualquier persona que labore en la empresa exceptuando al Dueño.\\

Estado: Hace referencia a los elementos, definiendo un elemento como: Cliente, Sucursal, Medicamento, Empleado, Paquete de descuento y Proveedor, Donde dicho elemento puede ser manipulado en operaciones de lectura y escritura (Estado activado) o solo lectura(Estado desactivado)\\

Ingrediente Activo: Sustancia del Medicamento con composición química exactamente conocida y que es capaz de producir efectos o cambios sobre el cuerpo de quien lo consume.\\

Laboratorio Farmacéutico: Aquellas personas físicas o jurídicas que, previamente autorizadas por la Administración competente, fabriquen de forma industrial Medicamentos o participen en alguna de sus fases.\\

Lote: Es una clave de identificación de los Medicamentos de un mismo proceso de fabricación.Tiene valor para el Laboratorio Farmacéutico. Consta de una letra , que indica el año de fabricación, y de un número.\\

Medicamento: Producto que sirve para curar, prevenir una enfermedad, para reducir sus efectos sobre el organismo o para aliviar un dolor físico.\\

Paquete De Descuento: Son paquetes que añade el dueño cuando se esta por caducar un medicamento, o es temporada en la que un medicamento tiene un rango de venta mayor que otros y se aprovecha para hacer un descuento de esos medicamentos, al conjunto de medicamentos que se van a poner en descuento es un: paquete de Descuento.\\

Sucursal: Una extension de la farmacia donde se opera como la farmacia original.
Esta Sucursal tiene nombre Único \\

Supervisor: (es un tipo de Empleado) Es un empleado con mayor Jerarquía que Empleado pero menor que Dueño.\\

Venta: Son los datos que se guardan de una venta de medicamento realizada.\\

Vía de Administración:  Es la forma en la que el cliente se tiene que aplicar el medicamento.entre estas están: Vía Digestiva, Vía Oral, Vía Sublingual, Gastroenteritis, Vía Rectal, Vía Parental, Vía Respiratoria, Vía Tópica, Vía Oftálmica, Vía Ótica, Vía Transdérmica.\\


Monedero electrónico: Dinero abonado por una devolución a la cuenta del cliente preferente.

\section{Hechos y términos del Negocio}
\section{Modelo de Dominio del problema}
\section{Reglas del Negocio}
\section{Procesos del Negocio}













