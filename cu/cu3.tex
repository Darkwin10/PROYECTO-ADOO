\begin{UseCase}{CU3}{Modificar datos de  Empleado}{
		Los datos del empleado se tienen que modificar, por inconsistencia en los datos registrados en el sistema con sus datos actuales
	}
		\UCitem{Versión}{\color{Gray}0.1.2}
		\UCitem{Autor}{\color{Gray}Aguilera Rosas Landa Enrique}
		\UCitem{Supervisa}{\color{Gray}}
		\UCitem{Actor}{\hyperlink{Alumno}{Dueño}}
		\UCitem{Propósito}{Evitar problemas con los empleados por datos erróneos guardados en el sistema.}
		\UCitem{Entradas}{Nombre completo del solicitante, Edad, Dirección, Curp, RFC, Experiencia Laboral, Puesto , Sucursal}
		\UCitem{Origen}{Teclado}
		\UCitem{Salidas}{No Aplica.}
		\UCitem{Destino}{Pantalla}
		\UCitem{Precondiciones}{El empleado debe de estar dado de alta en el sistema.}
		\UCitem{Postcondiciones}{Los datos del empleado serán diferentes.}
		\UCitem{Errores}{Exista algún duplicado en los datos del empleado, sus cambios no sean guardados}
		\UCitem{Observaciones}{En las Entradas: La experiencia laboral no es un atributo de los empleados, no se como se toma en cuenta ese parametro en el sistema.,paso 1: no exitende del CU0 lo incluye.paso 6: no hay botón que diga modificar empleado en las opciones,paso 7,8:no genera una tabla, muestra el formulario pero con los campos llenos ese formulario se encuentra en slack seccion de #modulo-de-interacción, el boton es {Actualizar} en lugar de {aceptar y guardar} solo existe una pantalla principal, no hay de dueño,empleado o supervisor, solo se controlan las opciones que se pueden hacer pero la pantalla es la misma.}
		\UCitem{Estado}{Corrección}
	\end{UseCase}
%--------------------------------------
	\begin{UCtrayectoria}{Principal}
		\UCpaso Se extiende del caso de uso \UCref{CU0} paso 11
		\UCpaso[\UCactor] Selecciona La opción de ver Lista de  Empleados presionando el botón \IUbutton{Ver Lista de Empleado}.
		\UCpaso Incluye el caso de uso \UCref{CU5}.
		\UCpaso[\UCactor] Introduce el Nombre del empleado a buscar en el campo de Búsqueda \Trayref{A} .
		\UCpaso Genera y Despliega una lista que coincida con la búsqueda realizada
		\UCpaso[\UCactor] Selecciona la opción modificar datos Del empleado Deseado presionando\IUbutton{Modificar Datos}.
		\UCpaso Genera y despliega una tabla con los datos actuales del empleado.
		\UCpaso Genera el formulario Datos del empleado y los despliega.
		\UCpaso[\UCactor] Cambia los datos que el empleado necesita modificar y guarda los cambios presionando el botón \IUbutton{Aceptar y Guardar} \Trayref{B}.
		\UCpaso Redirige al [\UCactor] a la  \IUref{01}{Pantalla Principal de Dueño}.
	\end{UCtrayectoria}


%-------------------------------------------------------------------------


\begin{UCtrayectoriaA}{A}{Empleado no encontrado.}
			\UCpaso Muestra el Mensaje {\bf MSG8-}`` [{\em Empleado no encontrado}] revisa que los campos sean llenados correctamente.''.
			\UCpaso Continúa en el paso \ref{CU7Bus} del \UCref{CU3}.
		\end{UCtrayectoriaA}
%-------------------------------------------------------------------------


\begin{UCtrayectoriaA}{B}{Algún campo del Empleado tiene un error.}
			\UCpaso Muestra el Mensaje {\bf MSG7-}`` [{\em Error en un dato del Empleado}] revisa que los campos sean llenados correctamente.''.
			\UCpaso Continúa en el paso 9 del \UCref{CU3}.
		\end{UCtrayectoriaA}
