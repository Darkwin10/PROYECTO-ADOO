\begin{UseCase}{CU6}{Reasignar Sucursales a Supervisor}{
		Un supervisor tiene la responsabilidad de checar el funcionamiento adecuado de las sucursales y por lo tanto se le asignaran nuevas sucursales dependiendo de los movimientos operacionales de cada una
	}
		\UCitem{Versión}{\color{Gray}0.1}
		\UCitem{Autor}{\color{Gray}Aguilera Rosas Landa Enrique}
		\UCitem{Supervisa}{\color{Gray}Correa Medican Carlos Miguel}
		\UCitem{Actor}{\hyperlink{Alumno}{Dueño}}
		\UCitem{Propósito}{Mejorar las operaciones diarias de la farmacia mediante resignaciones de supervisores adecuados.}
		\UCitem{Entradas}{Nombre del Empleado, Id de Empleado}
		\UCitem{Origen}{Teclado}
		\UCitem{Salidas}{No Aplica.}
		\UCitem{Destino}{Pantalla}
		\UCitem{Precondiciones}{El supervisor debe de estar registrado en el sistema y con un estado activado.}
		\UCitem{Postcondiciones}{ El supervisor tendrá otra sucursal diferente a la original .}
		\UCitem{Errores}{La pagina sea inaccesible por el momento debido a fallas con los servidores, Que el empleado tenga su cuenta desactivada}
		\UCitem{Observaciones}{en descripción El supervisor solo se encarga solo de las sucursales que se le asignan, {funcionamiento adecuado de las sucursales} se puede entender de que un supervisor esta acarga de todas las sucursales. postCondiciones:por lo menos una sucursal diferente, pero puede cambiar más de una sucursal. referencia a las IUX}
		\UCitem{Estado}{En Corrección}
	\end{UseCase}
%--------------------------------------
	\begin{UCtrayectoria}{Principal}
		\UCpaso Se extiende del caso de uso \UCref{CU3} paso 8
		\UCpaso [\UCactor] Modifica la sucursal del empleado y guarda los cambios presionando el botón\IUbutton{Aceptar y Guardar}
		\UCpaso Redirige al [\UCactor] a la  \IUref{01}{Pantalla Principal de Dueño}.
	\end{UCtrayectoria}


%-------------------------------------------------------------------------
\begin{UCtrayectoriaA}{A}{Empleado no encontrado.}
			\UCpaso Muestra el Mensaje {\bf MSG8-}`` [{\em Empleado no encontrado}] revisa que los campos sean llenados correctamente.''.
			\UCpaso Continúa en el paso 4 del \UCref{CU4}.
		\end{UCtrayectoriaA}
%-------------------------------------------------------------------------
