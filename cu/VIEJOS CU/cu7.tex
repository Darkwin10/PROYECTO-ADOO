\begin{UseCase}{CU7}{Modificar datos de  Empleado}{
		Los datos del empleado se tienen que modificar, por inconsistencia en los datos registrados en el sistema con sus datos actuales
	}
		\UCitem{Versión}{\color{Gray}0.1.2}
		\UCitem{Autor}{\color{Gray}Aguilera Rosas Landa Enrique}
		\UCitem{Supervisa}{\color{Gray}Correa Medina Carlos Miguel}
		\UCitem{Actor}{\hyperlink{Alumno}{Dueño}}
		\UCitem{Propósito}{Evitar problemas con los empleados por datos erroneos guardados en el sistema.}
		\UCitem{Entradas}{Nombre completo del solicitante, Edad, Direccion, Curp, RFC, Expreiencia Laboral, Puesto , Sucursa}
		\UCitem{Origen}{Teclado}
		\UCitem{Salidas}{No Aplica.}
		\UCitem{Destino}{Pantalla}
		\UCitem{Precondiciones}{El empleado debe de estar dado de alta en el sistema.}
		\UCitem{Postcondiciones}{Los datos del empleado seran diferentes.}
		\UCitem{Errores}{Exista algun duplicado en los datos del empleado, sus cambios no sean guardados}
		\UCitem{Tipo}{Caso de uso primario}
		\UCitem{Observaciones}{}
		\UCitem{Estado}{Aprobado}
	\end{UseCase}
%--------------------------------------
	\begin{UCtrayectoria}{Principal}
		\UCpaso Se extiende del caso de uso \UCref{CU0} paso 11
		\UCpaso[\UCactor] Selecciona La opcion de ver Lista de  Empleados presionando el boton \IUbutton{Ver Lista de Empleado}.
		\UCpaso Incluye el caso de uso \UCref{CU35}.
		\UCpaso[\UCactor] Introduce el Nombre del empleado a buscar en el campo de Busqueda \Trayref{A} \label{CU7Bus}.
		\UCpaso Genera y Despliega una lista que coincida con la busqueda realizada
		\UCpaso[\UCactor] Selecciona la opcion modificar datos Del empleado Deseado presionando\IUbutton{Modificar Datos} \label{CU7Cambio}.
		\UCpaso Genera y despliega una tabla con los datos actuales del empleado		
		\UCpaso Genera el formulario Datos del empleado y los despliega 
		\UCpaso[\UCactor] Cambia los datos que el empleado necesita modificar y guarda los cambios presionando el boton \IUbutton{Aceptar y Guardar} \Trayref{B}.
	\end{UCtrayectoria}


%-------------------------------------------------------------------------


\begin{UCtrayectoriaA}{A}{Empleado no encontrado.}
			\UCpaso Muestra el Mensaje {\bf MSG8-}`` [{\em Empleado no encontrado}] revisa que los campos sean llenados correctamente.''.
			\UCpaso Continúa en el paso \ref{CU7Bus} del \UCref{CU7}.
		\end{UCtrayectoriaA}
%-------------------------------------------------------------------------


\begin{UCtrayectoriaA}{B}{Algun campo del Empleado tiene un error.}
			\UCpaso Muestra el Mensaje {\bf MSG7-}`` [{\em Error en un dato del Empleado}] revisa que los campos sean llenados correctamente.''.
			\UCpaso Continúa en el paso \ref{CU7Cambio} del \UCref{CU7}.
		\end{UCtrayectoriaA}


