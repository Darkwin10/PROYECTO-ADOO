\begin{UseCase}{CU5}{Listar Empleados y sus Datos}{
		La farmacia consta con múltiples empleados y para poder checar los datos de todos los empleados, la opción mas eficaz es hacer un listado con los empleados y sus datos correspondientes
	}
		\UCitem{Versión}{\color{Gray}0.1.2}
		\UCitem{Autor}{\color{Gray}Aguilera Rosas Landa Enrique}
		\UCitem{Supervisa}{\color{Gray}Correa Medina Carlos Miguel}
		\UCitem{Actor}{Dueño}
		\UCitem{Propósito}{Control rápido sobre el manejo de los datos de los empleados.}
		\UCitem{Entradas}{Nombre del Empleado, Id de Empleado}
		\UCitem{Origen}{Teclado}
		\UCitem{Salidas}{No Aplica.}
		\UCitem{Destino}{Pantalla}
		\UCitem{Precondiciones}{El empleado debe de estar registrado en el sistema.}
		\UCitem{Postcondiciones}{El dueño checara la lista de los empleados y sus respectivos datos .}
		\UCitem{Errores}{La pagina sea inaccesible por el momento debido a fallas con los servidores, Que el empleado tenga su cuenta no este registrado}
		\UCitem{Observaciones}{la ultima parte del Atributo Errores del caso de uso es confusa,paso 1: no se extiende del CU0 paso 2: no hay boton {ver lista de Empleados} del paso 3 en adelante el caso de uso se torna confuso, se debe de especificar los botones de {supervisor} y {Cajero} que estan dentro del boton {Empleados} en el escritorio}
		\UCitem{Estado}{Corrección}
	\end{UseCase}
%--------------------------------------
	\begin{UCtrayectoria}{Principal}
		\UCpaso Se extiende del caso de uso \UCref{CU0} paso 11
		\UCpaso[\UCactor] Selecciona La opción de ver Lista de  Empleados presionando el botón \IUbutton{Ver Lista de Empleado}.
		\UCpaso[\UCactor] Introduce el Nombre del empleado a buscar en el campo de Búsqueda.
		\UCpaso Genera y Despliega una lista que coincida con la búsqueda realizada. \Trayref{A} 
		\UCpaso [\UCactor]Realiza las acciones necesarias y  Confirma la operación presionando el \IUbutton{Aceptar}
		\UCpaso Redirige al [\UCactor] a la  \IUref{01}{Pantalla Principal de Dueño}.
	\end{UCtrayectoria}


%-------------------------------------------------------------------------
\begin{UCtrayectoriaA}{A}{Empleado no encontrado.}
			\UCpaso Muestra el Mensaje {\bf MSG8-}`` [{\em Empleado no encontrado}] revisa que los campos sean llenados correctamente.''.
			\UCpaso Continúa en el paso 4 del \UCref{CU5}.
		\end{UCtrayectoriaA}
%-------------------------------------------------------------------------
