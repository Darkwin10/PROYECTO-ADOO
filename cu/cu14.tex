\begin{UseCase}{CU14}{Recibir Medicamento}{
		Si el proveedor llega con un nuevo lote de medicamentos el cajero debe de tener una opción para poder ingresar esos medicamentos, Los medicamentos que recibe el cajero se registraran en la opción de ingresos en la \IUref{01}{Pantalla Principal}. lo cual nos desplegara la \IUref{20}{Formulario Ingreso} y en esta pantalla ingresaremos los datos del Medicamento que recibimos, las unidades que se reciben y el lote del medicamento
	}
		\UCitem{Versión}{\color{Gray}0.1.4}
		\UCitem{Autor}{\color{Gray}Aguilera Rosas Landa Enrique}
		\UCitem{Supervisa}{\color{Gray}Zamora Gachuz Jesús Felipe}
		\UCitem{Actor}{Cajero}
		\UCitem{Propósito}{Control rápido y eficaz sobre el inventario de la farmacia.}
		\UCitem{Entradas}{Lote, Proveedor, Nombre del Medicamento, Unidades a Recibir del Medicamento}
		\UCitem{Origen}{Teclado, Lector de Código de Barras}
		\UCitem{Salidas}{Lote,  ID del Cajero, Nombre de la Sucursal, Fecha y Hora, ID del Medicamento, Unidades a Recibir del Medicamento, Total a Pagar.}
		\UCitem{Destino}{Pantalla}
		\UCitem{Precondiciones}{El Medicamento debe estar registrado en el sistema.}
		\UCitem{Postcondiciones}{Las unidades del Medicamento deben de aumentar de acuerdo a las unidades ingresadas.}
		\UCitem{Errores}{La pagina sea inaccesible por el momento debido a fallas con los servidores, Que el empleado tenga su cuenta no este registrado}
		\UCitem{Observaciones}{}
		\UCitem{Estado}{Revisión}
		\UCitem{Viene de}{CU0}
	\end{UseCase}
%--------------------------------------
	\begin{UCtrayectoria}{Principal}
		\UCpaso Incluye el caso de uso \UCref{CU0} 
		\UCpaso[\UCactor] Selecciona La opción Compras en la \IUref{01}{Pantalla Principal} presionando el botón \IUbutton{Compras}.
		\UCpaso Despliega las opciones de Compras en la \IUref{01}
		\UCpaso [\UCactor] Selecciona La opción Ingresos en la \IUref{01}{Pantalla Principal} presionando el botón \IUbutton{Ingresos}.
		\UCpaso Genera y Despliega la \IUref{20}{Formulario Ingreso} con los campos vacíos y listos para llenar.
		\UCpaso Llena el campo ID del cajero con el ID del cajero actualmente en sesión.
		\UCpaso Llena el campo Nombre de la Sucursal con el Nombre de la sucursal actualmente en sesión.
		\UCpaso Llena el campo Fecha y Hora con la fecha y hora actual del sistema.
		\UCpaso Genera una lista de los proveedores actualmente registrados en el sistema, sera mostrado mediante un Combobox.
		\UCpaso[\UCactor] Selecciona el proveedor de la lista generada en el Combobox. \Trayref{A}
		\UCpaso[\UCactor] Introduce el Lote del medicamento que recibe, Nombre del medicamento, unidades que recibe del medicamento. \Trayref{B}
		\UCpaso Verifica que el Nombre del medicamento Ingresado Exista en la lista de medicamentos registrados actualmente en el sistema. \Trayref{C}
		\UCpaso Calcula el total a pagar mediante la cantidad de unidades que se reciben del medicamento multiplicado por el precio de compra del medicamento
		\UCpaso[\UCactor] Guarda el medicamento que recibió presionando el botón \IUbutton{Guardar}
		\UCpaso Aumenta las unidades de medicamento que acaba de recibir en la tabla de medicamentos
		\UCpaso Redirige al [\UCactor] a la  \IUref{01}{Pantalla Principal}.
	\end{UCtrayectoria}


%-------------------------------------------------------------------------
\begin{UCtrayectoriaA}{A}{Proveedor no encontrado.}
			\UCpaso Muestra el Mensaje {\bf MSG01-}``Error en la Operación [{\em Proveedor no encontrado}] revisa que los campos sean llenados correctamente en la \IUref{20}{Formulario Ingreso}.''.
			\UCpaso Continúa en el paso 6 del \UCref{CU14}.
		\end{UCtrayectoriaA}
%-------------------------------------------------------------------------
\begin{UCtrayectoriaA}{B}{Proveedor no encontrado.}
			\UCpaso Muestra el Mensaje {\bf MSG01-}``Error en la Operación [{\em Error en Operación}] revisa que los campos sean llenados correctamente en la \IUref{20}{Formulario Ingreso}.''.
			\UCpaso Continúa en el paso 6 del \UCref{CU14}.
		\end{UCtrayectoriaA}
%-------------------------------------------------------------------------
\begin{UCtrayectoriaA}{C}{Medicamento no encontrado.}
			\UCpaso Muestra el Mensaje {\bf MSG01-}``Error en la Operación [{\em Medicamento no encontrado}] revisa que el medicamento ingresado exista en el inventario del sistema \IUref{20}{Formulario Ingreso}.''.
			\UCpaso Continúa en el paso 6 del \UCref{CU14}.
		\end{UCtrayectoriaA}
%-------------------------------------------------------------------------

